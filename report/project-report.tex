\documentclass[12pt,a4paper]{report}
\usepackage{graphicx}
\usepackage{amsmath}
\usepackage{hyperref}

\title{Peer-to-Peer File Sharing System}
\author{Yugal Khanal 2302704}
\date{}

\begin{document}

\maketitle

\begin{abstract}
    % Briefly summarize the project, objectives, methods, and outcomes (100-300 words).
    % This section should highlight the main points of the report, including the purpose of the project, the approach taken, and the key results achieved.

    This project aims to develop a decentralised peer-to-peer file sharing system with robustness, fault tolerance, stability, reliability and data security.
\end{abstract}

\tableofcontents
\newpage

\chapter{Introduction}
    % Outline the context of P2P file sharing and its relevance. Identify stakeholders and state the project’s aims.
    Provide a background of P2P file sharing systems, their uses, and importance. Include the project’s goals and key objectives.

\chapter{Research}
    % Conduct a literature review on existing P2P systems, network security in distributed systems, and fault tolerance.
    % Discuss how your approach improves upon current models.
    Review relevant literature, current solutions, and techniques in P2P file sharing and network security. Highlight gaps in existing systems that your project addresses.

\chapter{Legal, Social, Ethical, and Professional Issues}
    % Address issues such as data privacy laws, potential misuse, and ethical considerations.
    Discuss relevant legal, ethical, and professional issues, such as data privacy regulations, intellectual property concerns, and ethical use cases.


\chapter{System Requirements}
    % Detail functional and non-functional requirements, breaking down file-sharing, encryption, and network resilience needs.
    Outline the functional requirements (core features like file transfer, encryption) and non-functional requirements (usability, performance).

\chapter{Design}
    % Describe the overall architecture, including modules for peer discovery, encryption, file transfer, and UI/UX elements.
    Explain the design architecture of the system. Include high-level diagrams, module descriptions, and interactions among components.

\chapter{Implementation}
    % Document the development process, highlighting any challenges faced and key design decisions.
    Describe the implementation details, programming languages, libraries, and frameworks used. Highlight significant implementation choices and challenges encountered.

\chapter{Testing and Success Measurement}
    % Summarize the results of your testing phases, using quantitative data where possible.
    Discuss testing methodologies used (unit, integration, system tests). Include results and measures of success, such as performance metrics or security assessments.

\chapter{Project Management}
    % Discuss the project timeline, chosen development methodology, and major milestones.
    Provide a project timeline with phases and milestones. Explain the chosen development methodology (e.g., Agile, Waterfall) and justify its use.

\chapter{Evaluation}
    % Reflect on how well the project met the initial requirements and discuss areas where it excelled or fell short.
    Assess the project’s success relative to the requirements specified in Chapter 4. Discuss any limitations or areas for improvement.

\chapter{Conclusion}
    % Summarize key achievements, practical applications of the system, and possible future improvements.
    Summarize the project’s outcomes, practical applications, and potential areas for future work.

\chapter*{References}
    % Cite all resources consistently, following a preferred style (e.g., IEEE, APA).
    % Example:
    % \begin{thebibliography}{9}
    % \bibitem{latexcompanion} IEEE, "Title of paper", in Proc. of Conference, vol. X, no. X, pp. xx-xx, Year.
    % \end{thebibliography}
    List all references cited in the document, formatted consistently.

\appendix
\chapter{Appendix}
    % Include detailed code listings, additional diagrams, or supplementary testing data.
    Any supplementary information like code snippets, extended data tables, or additional diagrams can be placed here.

\end{document}
