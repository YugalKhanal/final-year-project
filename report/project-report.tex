\documentclass[12pt,a4paper]{report}
\usepackage{graphicx}
\usepackage{amsmath}
\usepackage{hyperref}
\usepackage{listings}
\usepackage{color}
\usepackage{float}
\usepackage{booktabs}

% Code listing settings
\definecolor{codegreen}{rgb}{0,0.6,0}
\definecolor{codegray}{rgb}{0.5,0.5,0.5}
\definecolor{codepurple}{rgb}{0.58,0,0.82}
\definecolor{backcolour}{rgb}{0.95,0.95,0.92}

\lstdefinestyle{mystyle}{
    backgroundcolor=\color{backcolour},
    commentstyle=\color{codegreen},
    keywordstyle=\color{magenta},
    numberstyle=\tiny\color{codegray},
    stringstyle=\color{codepurple},
    basicstyle=\ttfamily\footnotesize,
    breakatwhitespace=false,
    breaklines=true,
    captionpos=b,
    keepspaces=true,
    numbers=left,
    numbersep=5pt,
    showspaces=false,
    showstringspaces=false,
    showtabs=false,
    tabsize=2
}

\lstset{style=mystyle}

\title{Peer-to-Peer File Sharing System:\\A Robust and Scalable Implementation}
\author{Yugal Khanal\\2302704}
\date{\today}

\begin{document}

\maketitle

\begin{abstract}
	The proliferation of distributed systems has led to increased interest in peer-to-peer (P2P) architectures for file sharing. This dissertation presents the design, implementation, and evaluation of a robust P2P file sharing system that addresses key challenges in scalability, fault tolerance, and security. The system implements a hybrid architecture combining centralized tracking with distributed file storage, featuring chunked file transfer, piece verification, and concurrent downloading capabilities.

	The implementation includes sophisticated features such as tracker-based peer discovery, UPnP port mapping for NAT traversal, and a comprehensive piece management system for handling large file transfers. Through extensive testing and evaluation, the system demonstrates reliable performance under various network conditions while maintaining data integrity and transfer efficiency.

	This work contributes to the field by implementing novel approaches to common P2P challenges, including peer availability management and fault-tolerant file transfers, while providing insights into the practical considerations of building distributed systems.

	\textbf{Keywords:} Peer-to-Peer Networks, Distributed Systems, File Sharing, Network Programming, Fault Tolerance
\end{abstract}

\tableofcontents
\listoffigures
\listoftables

\chapter{Introduction}
\section{Background and Motivation}
 [Discussion of the evolution of P2P systems and their role in modern networking]

The increase is the amount of Peer-to-Peer (P2P) file-sharing systems has changed digital content distribution. It's main advantage over it's traditional counterpart client-server model where a central authority manages file transfers is that P2P networks distribute this responsibility over to the peers in the network significantly increasing scalability and fault tolerance.

Data consumption has had a boom in recent years and because of the limitations of the client-server architecture such as high bandwidth costs, central points of failure etc., there has been a demand for efficient, secure and scalable file-sharing mechanisms. Modern P2P networks are the perfect solution to this. They incorporate advanced networking protocols, cryptographic security, and optimisation algorithms to enhance their performance and security. Enhanced scalability is achieved by distributing the file storage and transfer responsibilities among the peers in the network. This reduces the load on individual nodes and improves the overall efficiency of the system. Data redundancy and availability are also improved by storing multiple copies of the file across the network. These two are very strong points for P2P networks and any data distribution system would benefit from these features.

P2P networks have wider applications beyond just simple file-sharing. It's applications range from content distribution networks (CDNs) to blockchains and distributed cloud computing. There are also significant challenges in this approach. Some of them being data security, peer reliability and performance bottlenecks. This project aims to address some of these challenges by designing and implementing a secure, scalable and fault-tolerant P2P file-sharing system.


\section{Project Objectives}
 [Clear enumeration of project goals and success criteria]

The main objective of this project is to design and implement a robust and scalable P2P file-sharing system that addresses key challenges in scalability, fault tolerance, and security. The system will be evaluated based on the following criteria:

\begin{itemize}
	\item Develop a decentralised file-sharing system: The system should be able to handle a large number of peers and files while maintaining performance.
	\item Fault Tolerance: The system should be resilient to peer failures and network disruptions, ensuring reliable file transfers.
	\item Security: The system should implement secure communication protocols and data encryption to protect user data.
	\item Performance: The system should provide efficient file transfer mechanisms and low latency for peer interactions.
	\item Implement secure encryption mechanisms to safeguard data integrity.
\end{itemize}

\section{Evaluation Criteria}
 [Description of the metrics and methods used to evaluate the system]

\begin{itemize}
	\item System Performance: The system's performance will be evaluated it terms of latency, download speeds, system response and throughput efficiency under various network conditions.
	\item Scalability: The system's ability to handle increasing numbers of simultaneous connections and file-sharing requests without performance degradation.
	\item Fault Tolerance: Evaluation of the system's resilience to peer failures and disconnections, network disruptions, and data corruption.
	\item Security: Effefctiveness of the implemented encryption algorithms, peer authentication mechanisms and safeguards against malicious attacks.
	\item Resource Utilization: The system's resource consumption in terms of CPU, memory, and network bandwidth. How efficient the system is in using bandwidth and system resources.
\end{itemize}

By evaluating the system based on these criteria, we can determine the effectiveness of the implemented features and the overall performance of the P2P file-sharing system. This project will hopefully also contribute to the advancement and highlight the importance of decentralized technologies.

\section{Problem Statement}
 [Detailed description of the challenges in P2P file sharing]

\section{Project Scope}
 [Outline of what the project encompasses and its boundaries]
Peer-to-peer file sharing is the distribuition

\chapter{Literature Review}
\section{History of P2P Systems}
 [Evolution of P2P architectures and protocols]

At the start, file-sharing was exclusively done by client-server

\section{BitTorrent Protocol Analysis}
 [Detailed examination of BitTorrent's approach]

\section{Modern P2P Applications}
 [Survey of current P2P implementations]

\section{Security Challenges in P2P Networks}
 [Analysis of security considerations]

\section{Distributed Hash Tables and Peer Discovery}
 [Review of peer discovery mechanisms]

\chapter{System Architecture}
\section{High-Level Design}
 [System overview with architectural diagrams]

\section{Component Overview}
 [Detailed description of system components]

\section{Network Protocol Design}
 [Protocol specifications and communication patterns]

\section{Data Flow Architecture}
 [Data flow diagrams and explanations]

\section{Storage System Design}
 [File storage and management architecture]

\chapter{Implementation Details}
\section{Tracker Implementation}
 [Details of the tracking system]

\section{Peer Discovery and Management}
 [Peer handling mechanisms]

\section{File Chunking and Transfer Protocol}
 [File transfer implementation details]

\section{Concurrent Download Management}
 [Concurrency handling approaches]

\section{Error Handling and Recovery}
 [Error management strategies]

\section{Security Implementation}
 [Security measures and protocols]

\chapter{Technical Challenges and Solutions}
\section{Network NAT Traversal}
 [NAT handling implementation]

\section{File Integrity Verification}
 [Data verification mechanisms]

\section{Peer Availability Management}
 [Peer management strategies]

\section{Performance Optimization}
 [Performance improvements]

\section{Fault Tolerance Implementation}
 [Fault handling approaches]

\chapter{Testing and Evaluation}
\section{Performance Metrics}
 [Performance testing results]

\section{Scalability Testing}
 [Scalability analysis]

\section{Network Resilience}
 [Network testing results]

\section{Security Testing}
 [Security evaluation]

\section{User Experience Testing}
 [Usability assessment]

\chapter{Conclusion and Future Work}
\section{Project Achievements}
 [Summary of accomplishments]

\section{Limitations}
 [Project limitations]

\section{Future Improvements}
 [Potential enhancements]

\section{Final Reflections}
 [Concluding thoughts]

\bibliographystyle{ieeetr}
\bibliography{references}

\appendix
\chapter{Code Listings}
\section{Core Components}
% Example code listing:
\begin{lstlisting}[language=Go, caption=Tracker Implementation]
type Tracker struct {
    fileIndex    map[string]*FileInfo
    peerIndex    map[string]map[string]bool
    peerLastSeen map[string]time.Time
    mu           sync.RWMutex
}
\end{lstlisting}

\chapter{Testing Data}
 [Detailed test results and analysis]

\chapter{User Manual}
 [System usage instructions]


\end{document}
